%----------------------------------------------------------------------------------------
%
% A LaTeX-template for 1DV510. Modified and translated by Björn Lindenberg at LNU.
% Based on an original master thesis template created by Marcus Wilhelmsson at LNU.
%
%----------------------------------------------------------------------------------------

% Settings and document configuration

\documentclass[a4paper,12pt]{article} 
\usepackage[T1]{fontenc} 
\usepackage{times} 
\usepackage[swedish,english]{babel} 
\usepackage[utf8]{inputenc} 
\usepackage{dtk-logos} 
\usepackage{wallpaper} 
\usepackage[absolute]{textpos} 
\usepackage[top=2cm, bottom=2.5cm, left=3cm, right=3cm]{geometry} 
\usepackage[parfill]{parskip} 
\usepackage{csquotes} 
\usepackage{float} 
\usepackage{lipsum} % Used for dummy text. Can be removed.
\usepackage[nottoc]{tocbibind}
% Fontsizes for section headings.
\usepackage{sectsty} 
\sectionfont{\fontsize{14}{15}\selectfont}
\subsectionfont{\fontsize{12}{15}\selectfont}
\subsubsectionfont{\fontsize{12}{15}\selectfont}

%----------------------------------------------------------------------------------------
%	This part is used for the text box on the title page
%----------------------------------------------------------------------------------------
\newsavebox{\mybox}
\newlength{\mydepth}
\newlength{\myheight}

\newenvironment{sidebar}%
{\begin{lrbox}{\mybox}\begin{minipage}{\textwidth}}%
{\end{minipage}\end{lrbox}%
 \settodepth{\mydepth}{\usebox{\mybox}}%
 \settoheight{\myheight}{\usebox{\mybox}}%
 \addtolength{\myheight}{\mydepth}%
 \noindent\makebox[0pt]{\hspace{-20pt}\rule[-\mydepth]{1pt}{\myheight}}%
 \usebox{\mybox}}

%----------------------------------------------------------------------------------------
%	Title
%----------------------------------------------------------------------------------------
\newcommand\BackgroundPic{
    \put(-2,-3){
    \includegraphics[keepaspectratio,scale=0.3]{img/lnu_etch.png} % Background image
    }
}
\newcommand\BackgroundPicLogo{
    \put(30,740){
    \includegraphics[keepaspectratio,scale=0.10]{img/logo.png} % LNU logo
    }
}

\title{
\vspace{-8cm}
\begin{sidebar}
    \vspace{10cm}
    \normalfont \normalsize
    \huge Report\\ % Main title
    \vspace{-1.3cm}
\end{sidebar}
\vspace{3cm}
\begin{flushleft}
    \huge Seminar 2: Cybercrime Law% Subtitle
\end{flushleft}
\null
\vfill
\begin{textblock}{6}(10,13)
\begin{flushright}
\begin{minipage}{\textwidth}
\begin{flushleft} \large
\emph{Author:}  Anas \textsc{Kwefati}\\  % Author
\emph{Email:} ak223wd@student.lnu.se\\ %Email
\emph{Semester:} Spring 2021\\ % Semester
\emph{Area:} Digital Forensics \\ % Area
\emph{Course code:} 2DV704 % Course
\end{flushleft}
\end{minipage}
\end{flushright}
\end{textblock}
}

\date{} % Empty date command. Use \today inside for today's date.
\author{} % Normally one would use this to define authors. However in this case the title command takes care of everything, so we leave the field empty to get rid of warnings. 

\begin{document}

\pagenumbering{gobble} % Turn off page numbering
\newgeometry{left=5cm}
\AddToShipoutPicture*{\BackgroundPic} % Adds the background image to the title page
\AddToShipoutPicture*{\BackgroundPicLogo} % Adds the logo to the title page
\maketitle % Prints the title
\restoregeometry
\clearpage

\pagenumbering{roman} % Roman page numbering for abstract page


\newpage

\pagenumbering{gobble} % Turn off page numbering
\tableofcontents 

\newpage
\pagenumbering{arabic} % Turn on page numbering

% Some example sections with dummy text
\section{Question 1}
In the Budapest convention, it is said in Chapter 2 Section 1 Article 6, that, when someone uses a device or a computer program, namely a hacking tool, without having the intent to commit a cybercrime, it is considered in that situation as fine to possess such tool. However, if we do a crime, the use of such tools will be considered as a crime \cite{budapest}. Therefore, I am wondering, what would happen, if for example, someone is using such tools for education purposes, and did manage to break a network security without being aware of it, or without doing it on purpose. 

Would it be considered as a crime in that case (which may also break other articles, such as Article 2 - Illegal access)? Or is there any law that protects someone in that case? As the first intent, was for learning purposes, and not to commit a crime. 


%we have hacking tools, and we didn’t do any crime, it is not considered as crime. But if we do a crime, it’ll be considered as a crime. What would happen, if for instance, someone is just learning how to use tool on its own network, but then somehow, managed to violate the network security, would it be considered as a crime in that case? As the first intent, was to learn for educational purposes, but not really to steal or destroy something? 1:39:45


\section{Question 2}
The Cybercrime convention, has been signed and ratified by a certain number of countries. What would happen if an attacker is located in a country that did not sign, then manages to get access in a computer located in a country that signed the convention, and somehow starts to hack from that computer to other computers in the network, and maybe more. The victim's computer, is not aware of this fact, until the police show up to the person. 

How would law help this person to prove innocent, and how will they manage to cooperate with a country that did not sign the convention with the idea that they manage to find where the real attacker is located? 

In the book, it is said, that "agencies have to cooperate internationally in order to secure and exchange evidence" \cite{book1}, but what if the crime committed is not considered illegal in the attacker country, thus, they do not want to cooperate? Or the country is in war and do not have time to cooperate? 

%In the cybercrime convention, the word hacking tool is left as ambiguous as it is referred as a device, which can include a computer program or any device to commit a crime. They seem to not want to define exactly what is a hacking tool, to be open to a wide range of tools, and to the fast changes in technology. But, according to the book in Chapter 3, it is said that different interpretation of word can occur \cite{book}. In that way, what would happen, if someone create a hacking tool on a new technology and do crime with this tool. 
%
%
%
%In the cybercrime convention, the word hacking tool is left as ambiguous as they do not really define what is a hacking tool, to be open to a wide range of tools. But according to the book in chapter 3, it is said that different interpretation of a word can occur. So, what would happen if for instance, someone create actually a hacking tool for a new technology and somehow with lawyer’s help, manage to create an interpretation of that tool that is quite different from the reality? Wouldn’t it be a flaw in a system? And even thought investigators know about the hacking tool, they can’t come up with enough evidence to stop that issue. 

\section{Question 3}
If someone uses my network, to commit illegal activities, therefore, I am being charged on something that I have not committed, even though I am trying to prove my innocence, they do not believe it. So, I decide to hire a professional hacker, that would help me find real evidence that can prove my innocence, but unfortunately, the information to prove my innocence can only be obtained illegally. 

How would the court use information obtained illegally, and that is the only way at the moment to prove this person innocent? 

%What if someone managed to obtain relevant information to case, but illegally. How would a court use that information? For instance, someone is trying to hack my network, or network…, and with my extraordinary hacking skills that I have learnt at LNU, I managed to identify that person, with real proofs. How would the law work in that way? 
%parallel development -> evidence obtained illegally and inadmissible 
%fruit of poison tree
% Prints your bibliography database xxx.bib
\bibliographystyle{IEEEtran}
\bibliography{ref.bib}

\end{document}
